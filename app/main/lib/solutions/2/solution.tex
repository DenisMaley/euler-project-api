\documentclass[12pt]{article}
\usepackage{fontspec}
\usepackage{polyglossia}
\usepackage{empheq}
\usepackage{amssymb}
\usepackage{mathtools}
\usepackage{xcolor}
\usepackage{amsmath}
\usepackage{array}% http://ctan.org/pkg/array
\setdefaultlanguage{english}
\setmainfont[Mapping=tex-text]{CMU Serif}
\DeclarePairedDelimiter\floor{\lfloor}{\rfloor}

\title{Even Fibonacci numbers}
\author{
Denis Maley  \\
Amsterdam  \\
}

\date{\today}

\begin{document}

\maketitle

\textbf{\large The problem description:}

\bigskip

Each new term in the Fibonacci sequence is generated by adding the 
previous two terms. 
By starting with 0 and 1, the first 12 terms will be:

$$0,\> 1,\> 1,\> 2,\> 3,\> 5,\> 8,\> 13,\> 21,\> 34,\> 55,\> 89,\> \ldots$$

By considering the terms in the Fibonacci sequence whose values do 
not exceed four million, find the sum of the even-valued terms.

\bigskip

Let's reformulate the problem in general form:

\bigskip

\textit{
Given a non-negative integer $n$, find the sum of the even-valued terms 
in the Fibonacci sequence whose values are below $n$
}

\bigskip


\newpage

\textbf{\large Solution:}

\bigskip    



We can notice notice and easily prove that every third Fibonacci number is even:

$$
\textit{\color{red} 0},\> 1,\> 1,\> 
\textit{\color{red} 2},\> 3,\> 5,\> 
\textit{\color{red} 8},\> 13,\> 21,\> 
\textit{\color{red} 34},\> 55,\> 89,\> 
\textit{\color{red} 144},\> 233,\> 377,\> \ldots
$$

The proof easily follows from the fact that the sum is even 
if both terms are even or odd:

$$2k + 2l = 2(k+l)$$
$$(2k+1) + (2l+1) = 2(k+l+1)$$

\bigskip

If we only write the even numbers: 

$$0,\> 2,\> 8,\> 34,\> 144,\> \ldots$$

we can prove that they obey the following recursive relation: 

$$E(n)=4E(n-1)+E(n-2)$$

If we can prove that for the Fibonacci numbers the formula 

$$F(n)=4F(n-3)+F(n-6)$$ 
holds we have proven this recursion and we can easily program it.

\bigskip

Proof:

\begin{equation*}
\begin{split}
F(n) & = F(n-1) + F(n-2) \\
& = F(n-2)+F(n-3)+F(n-2) \\
& = 2F(n-2) + F(n-3) \\
& = 2(F(n-3)+F(n-4))+F(n-3) \\
& = 3F(n-3) + 2F(n-4) \\
& = 3F(n-3) + F(n-4) + F(n-4) \\
& = 3F(n-3) + F(n-4) + F(n-5) + F(n-6) \\
& = 4F(n-3) + F(n-6)
\end{split}
\end{equation*}


\bigskip
\end{document}
